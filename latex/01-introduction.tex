\mainmatter

# (PART\*) Part I: Foundations {-}

# Introduction

In recent years, there has been an exponential growth in the amount of data available for computational analysis, including scientific data as well as common data types such as text, images, and audio. This abundance of data has empowered various fields, including physics, chemistry, computational social sciences, and biology, to make significant progress using machine learning techniques, primarily deep neural networks. As deep neural networks can effectively summarize and extract patterns from large data sets, they are suitable for many complex tasks. Initially, deep neural networks have been developed to learn from data supported on regular (Euclidean) domains,
such as grids in images, sequences of text and time-series. These models, including convolutional neural networks (CNNs)~\cite{krizhevsky2012imagenet,lecun1998gradient,simonyan2014very}, recurrent neural networks (RNNs)~\cite{sutskever2014sequence,bahdanau2014neural} and transformers \cite{vaswani2017attention}, have proven highly effective in processing such Euclidean data~\cite{goodfellow2016deep}, resulting in unprecedented performance in various applications, most recently in chat-bots (e.g., ChatGPT~\cite{adesso2022gpt4}) and text-controlled image synthesis~\cite{rombach2022high}.

However, scientific data in various fields are often structured differently and are not supported on regular Euclidean domains. As a result, adapting deep neural networks to process this type of data has been a challenge. Against this backdrop, geometric deep learning (GDL)~\cite{bronstein2021geometric,zhou2020graph,wu2020comprehensive} has emerged as an extension of deep learning models to non-Euclidean domains. To achieve this, GDL restricts the performed computations via principles of geometric regularity, such as symmetries, invariances, and equivariances. The GDL perspective allows for appropriate inductive biases to be imposed when working with arbitrary data domains, including sets~\cite{qi2017pointnet,rempe2020caspr,deng2018ppfnet,zhao20223dpointcaps++,huang2022multiway}, grids~\cite{boscaini2015learning,masci2015geodesic,boscaini2016learning,kokkinos2012intrinsic,shuman2016vertex,wu20153d,monti2017geometric}, manifolds~\cite{boscaini2015learning,masci2015geodesic,boscaini2016learning,kokkinos2012intrinsic,shuman2016vertex,wu20153d,monti2017geometric}, and graphs~\cite{scarselli2008graph,gallicchio2010graph,zhou2020graph,wu2020comprehensive,boscaini2016learning,monti2017geometric,bronstein2017geometric,kipf2016semi}. 
Graphs, in particular, have garnered interest due to their applicability in numerous scientific studies and their ability to generalize conventional grids. Accordingly, the development of graph neural networks (GNNs)~\cite{bronstein2017geometric,kipf2016semi} has remarkably enhanced our ability to model and analyze several types of data in which graphs naturally appear.

Despite the success of GDL and GNNs, seeing graphs through a purely geometric viewpoint yields a solely local abstraction and falls short of capturing non-local properties and dependencies in data. \emph{Topological data}, including interactions of edges (in graphs), triangles (in meshes) or cliques, arise naturally in an array of novel applications in complex physical systems~\cite{battiston2021physics,lambiotte2019networks}, traffic forecasting~\cite{jiang2022graph}, social influence~\cite{zhu2018social}, protein interaction~\cite{murgas2022hypergraph}, molecular design~\cite{schiff2020characterizing}, visual enhancement~\cite{efthymiou2021graph}, recommendation systems \cite{la2022music}, and epidemiology~\cite{deng2020cola}.
To natively and effectively model such data, we are bound to go beyond graphs and consider qualitative spatial properties remaining unchanged under some geometric transformations. In other words, we need to consider the \emph{topology of data}~\cite{carlsson2009topology} to formulate neural network architectures capable of extracting semantic meaning from complex data.


One approach to extract more global information from data is to go beyond graph-based abstractions and consider extensions of graphs, such as simplicial complexes, cell complexes, and hypergraphs, generalizing most data domains encountered in scientific computations~\cite{bick2021higher,battiston2020networks,benson2021higher,torres2021and}.
The development of machine learning models to learn from data supported on these topological domains~\cite{feng2019hypergraph,bunch2020simplicial, roddenberry2021signal,schaub2020random,billings2019simplex2vec,hacker2020k,hajijcell,ebli2020simplicial,schaub2021signal,roddenberry2021principled,giusti2022cell,yang2023convolutional} is a rapidly growing new frontier, to which we refer hereafter as \emph{topological deep learning (TDL)}.
TDL intertwines several research areas, including topological data analysis (TDA)~\cite{edelsbrunner2010computational,carlsson2009topology,DW22,love2020topological,ghrist2014elementary}, topological signal processing~\cite{schaub2018denoising,yang2021finite,schaub2022signal,roddenberry2021signal,barbarossa2020topological,robinson2014topological,sardellitti2022topological}, network science~\cite{skardal2021higher,lambiotte2019networks,barabasi2013network,battiston2020networks,bick2021higher,bianconi2021higher,benson2016higher,de2016physics,bao2022impact,oballe2021bayesian}, and geometric deep learning~\cite{zhang2020deep,cao2020comprehensive,fey2019fast,loukas2019graph,battaglia2018relational,morris2019weisfeiler,battaglia2016interaction}.

Despite the growing interest in TDL, a broader synthesis of the foundational principles of these ideas has not been established so far. We argue that this is a deficiency that inhibits progress in TDL, as it makes it challenging to draw connections between different concepts, impedes comparisons, and makes it difficult for researchers of other fields to find an entry point to TDL. Hence, in this paper, we aim to provide a foundational overview over the principles of TDL, serving not only as a unifying framework for the many exciting ideas that have emerged in the literature over recent years, but also as a conceptual starting point to promote the exploration of new ideas. Ultimately, we hope that this work will contribute to the accelerated growth in TDL, which we believe would be a key enabler of transferring deep learning successes to an enlarged range of application scenarios.

By drawing inspiration from traditional topological notions in algebraic topology~\cite{ghrist2014elementary,hatcher2005algebraic} and recent advancements in higher-order networks~\cite{battiston2020networks,torres2021and,bick2021higher,battiston2021physics}, we first introduce \emph{combinatorial complexes (CCs)} as the major building blocks of our TDL framework. CCs constitute a novel topological domain that unifies graphs, simplicial and cell complexes, and hypergraphs as special cases, as illustrated in Figure~\ref{fig:main}\footnote{All figures in this paper should be displayed in color, as different colors communicate different pieces of information.}. 
Similar to hypergraphs, CCs can encode arbitrary set-like relations between collections of abstract entities. Moreover, CCs permit the construction of hierarchical higher-order relations, analogous to those found in simplicial and cell complexes. Hence, CCs generalize and combine the desirable traits of hypergraphs and cell complexes. 

In addition, we introduce the necessary operators to construct deep neural networks for learning features and abstract summaries of input anchored on CCs. Such operators provide convolutions, attention mechanisms, message-passing schemes as well as the means for incorporating invariances, equivariances, or other geometric regularities. Specifically, our novel \emph{push-forward operation} allows pushing data across different dimensions, thus forming an elementary building block upon which \emph{higher-order message-passing protocols} and \emph{(un)pooling operations} are defined on CCs. The resulting learning machines, which we call \emph{combinatorial complex neural networks (CCNNs)}, are capable of learning abstract higher-order data structures, as clearly demonstrated in our experimental evaluation. 

We envisage our contributions to be a platform encouraging researchers and practitioners to extend our $\CCN$s, and invite the community to build upon our work to expand TDL on higher-order domains.
Our contributions, which are also visually summarized in Figure~\ref{fig:main}, are the following:

\begin{itemize}[topsep=1pt]
\setlength\itemsep{0.1pt}
\item First, we introduce CCs as domains for TDL. We characterize CCs and their properties, and explain how
they generalize major existing domains, such as graphs, hypergraphs, simplicial and cell complexes. CCs can thus serve as a unifying starting point that enables the learning of expressive representations of topological data. 
\item Second, using CCs as domains, we construct $\CCN$s, an abstract class of higher-order message-passing neural networks that provide a unifying blueprint for TDL models based on hypergraphs and cell complexes. 
\begin{itemize}[topsep=1pt]
\setlength\itemsep{0.1pt}
    \item Based upon a push-forward operator defined on CCs, we introduce convolution, attention, pooling and unpooling operators for $\CCN$s. 
    \item We formalize and investigate \emph{permutation and orientation equivariance} of $\CCN$s, paving the way to future work on geometrization of $\CCN$s.
    \item We show how CCNNs can be intuitively constructed via graphical notation.
\end{itemize}
\item Third, we evaluate our ideas in practical scenarios.
\begin{itemize}[topsep=1pt,noitemsep]
    \item We release the source code of our framework as three supporting Python libraries: \emph{TopoNetX}, \emph{TopoEmbedX} and \emph{TopoModelX}.
    \item We show that $\CCN$s attain competing predictive performance against state-of-the-art task-specific neural networks in various applications, including shape analysis and graph learning. 
    \item We establish connections between our work and classical constructions in TDA, such as the \emph{mapper}~\cite{singh2007topological}. Particularly, we realize the mapper construction in terms of our TDL framework and demonstrate how it can be utilized in higher-order (un)pooling on CCs.
    \item  We demonstrate the reducibility of any CC to a special graph called the \emph{Hasse graph}. This enables the characterization of certain aspects of $\CCN$s in terms of graph-based models, allowing us to reduce higher-order representation learning to graph representation learning (using an enlarged computational graph).
\end{itemize}
\end{itemize}

\paragraph{Glossary.} Prior to delving into more details, we present the elementary terminologies of already established concepts used throughout the paper. Some of these terms are revisited formally in Section~\ref{prelim}. Appendix~\ref{appendix_glossary} provides notation and terminology glossaries for novel ideas put forward by the paper.

\vspace{-0.5mm}
\begin{tcolorbox}
 % \begin{multicols}{2}
% \setlist{nolistsep}
% \begin{itemize}
%\small{
\textbf{\href{https://app.vectary.com/p/3EBiRiJcYjFNvkbbWszQ0Z}{Cell complex}}\label{SX.term}: a topological space obtained as a disjoint union of topological disks (cells), with each of these cells being homeomorphic to the interior of a Euclidean ball. These cells are attached together via attaching maps in a locally reasonable manner. \\[1mm]
\textbf{Domain}\label{domain.term}: the underlying space where data is typically supported.\\[1mm]
\textbf{Entity} or \textbf{vertex}\label{entity.term}: an abstract point; it can be thought as an element of a set.\\[1mm]
\textbf{Graph} or \textbf{network}\label{G.term}: a set of entities (vertices) together with a set of edges representing binary relations between the vertices.\\[1mm]
\textbf{Hierarchical structure} on a topological domain: an integer-valued function that assigns a positive integer (rank) to every relation in the domain such that higher-order relations are assigned higher rank values. For instance, a simplicial complex admits a hierarchical structure induced by the cardinality of its simplices.\\[1mm]
\textbf{Higher-order network/topological domain}: generalization of a graph that captures (binary and) higher-order relations between entities. Simplicial/cell complexes and hypergraphs
are examples of higher-order networks.\\[1mm]
\textbf{Hypergraph}: a set of entities (vertices) and a set of hyperedges representing binary or higher-order relations between the vertices.\\[1mm]
\textbf{Message passing}: a computational framework that involves passing data, `messages', among neighbor entities in a domain to update the representation of each entity based on the messages received from its neighbors.\\[1mm]
\textbf{Relation} or \textbf{cell}\label{cell.term}: a subset of a set of entities (vertices).
A relation is called \textbf{binary} if its cardinality is equal to two.
A relation is called \textbf{higher-order} if its cardinality is greater than two.\\[1mm]
\textbf{Set-type relation}: a higher-order network is said to have set-type relations if the existence of a relation is not implied by another relation in the network; e.g., hypergraphs admit set-type relations.\\[1mm]  
\textbf{Simplex}: a generalization of a triangle or tetrahedron to arbitrary dimensions; e.g.,
a simplex of dimension zero, one, two, or three is a point, line segment, triangle, or tetrahedron, respectively.\\[1mm]
\textbf{\href{https://app.vectary.com/p/4HZRioKH7lZ2jWESIBrjhf}{Simplicial complex}}\label{SC.term}: a collection of simplices such that every face of each simplex in the collection is also in the collection, and the intersection of any two simplices in the collection is either empty or a face of both simplices.\\[1mm]
\textbf{Topological data}\label{TD.term}: feature vectors supported on relations in a topological domain.\\[1mm] 
\textbf{Topological deep learning}\label{TDL.term}: the study of topological domains using deep learning techniques, and the use of topological domains to represent data in deep learning models.\\ 
\textbf{Topological neural network}\label{TDL.term}: a deep learning model that processes data supported on a topological domain. 
 %   }
% \end{itemize}
% \end{multicols}
\end{tcolorbox}
